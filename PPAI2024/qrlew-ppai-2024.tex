%File: anonymous-submission-latex-2024.tex
\documentclass[letterpaper]{article} % DO NOT CHANGE THIS
\usepackage[submission]{aaai24}  % DO NOT CHANGE THIS
\usepackage{times}  % DO NOT CHANGE THIS
\usepackage{helvet}  % DO NOT CHANGE THIS
\usepackage{courier}  % DO NOT CHANGE THIS
\usepackage[hyphens]{url}  % DO NOT CHANGE THIS
\usepackage{graphicx} % DO NOT CHANGE THIS
\urlstyle{rm} % DO NOT CHANGE THIS
\def\UrlFont{\rm}  % DO NOT CHANGE THIS
\usepackage{natbib}  % DO NOT CHANGE THIS AND DO NOT ADD ANY OPTIONS TO IT
\usepackage{caption} % DO NOT CHANGE THIS AND DO NOT ADD ANY OPTIONS TO IT
\frenchspacing  % DO NOT CHANGE THIS
\setlength{\pdfpagewidth}{8.5in} % DO NOT CHANGE THIS
\setlength{\pdfpageheight}{11in} % DO NOT CHANGE THIS
%
% These are recommended to typeset algorithms but not required. See the subsubsection on algorithms. Remove them if you don't have algorithms in your paper.
\usepackage{algorithm}
\usepackage{algorithmic}

%
% These are are recommended to typeset listings but not required. See the subsubsection on listing. Remove this block if you don't have listings in your paper.
\usepackage{newfloat}
\usepackage{listings}
\DeclareCaptionStyle{ruled}{labelfont=normalfont,labelsep=colon,strut=off} % DO NOT CHANGE THIS
\lstset{%
	basicstyle={\footnotesize\ttfamily},% footnotesize acceptable for monospace
	numbers=left,numberstyle=\footnotesize,xleftmargin=2em,% show line numbers, remove this entire line if you don't want the numbers.
	aboveskip=0pt,belowskip=0pt,%
	showstringspaces=false,tabsize=2,breaklines=true}
\floatstyle{ruled}
\newfloat{listing}{tb}{lst}{}
\floatname{listing}{Listing}
%
% Keep the \pdfinfo as shown here. There's no need
% for you to add the /Title and /Author tags.
\pdfinfo{
/TemplateVersion (2024.1)
}

\setcounter{secnumdepth}{0} %May be changed to 1 or 2 if section numbers are desired.

% The file aaai24.sty is the style file for AAAI Press
% proceedings, working notes, and technical reports.
%

% Title

% Your title must be in mixed case, not sentence case.
% That means all verbs (including short verbs like be, is, using,and go),
% nouns, adverbs, adjectives should be capitalized, including both words in hyphenated terms, while
% articles, conjunctions, and prepositions are lower case unless they
% directly follow a colon or long dash
\title{\emph{Qrlew}: automatic differential privacy for SQL queries}
\author{
    %Authors
    % Authors
    Nicolas Grislain\textsuperscript{\rm 1}
    Paul Roussel\textsuperscript{\rm 1}
    Victoria de Sainte-Agathe\textsuperscript{\rm 1}
}
\affiliations{
    %Afiliations
    \textsuperscript{\rm 1}Sarus Technologies\\
    % If you have multiple authors and multiple affiliations
    % use superscripts in text and roman font to identify them.
    % For example,

    % Sunil Issar\textsuperscript{\rm 2},
    % J. Scott Penberthy\textsuperscript{\rm 3},
    % George Ferguson\textsuperscript{\rm 4},
    % Hans Guesgen\textsuperscript{\rm 5}
    % Note that the comma should be placed after the superscript

    1900 Embarcadero Road, Suite 101\\
    Palo Alto, California 94303-3310 USA\\
    % email address must be in roman text type, not monospace or sans serif
    proceedings-questions@aaai.org
%
% See more examples next
}

\iffalse
%Example, Multiple Authors, ->> remove \iffalse,\fi and place them surrounding AAAI title to use it
\title{My Publication Title --- Multiple Authors}
\author {
    % Authors
    Nicolas Grislain,
    Paul Roussel,
    Victoria de Sainte-Agathe,
}
\affiliations {
    % Affiliations
    ng@sarus.tech, pr@sarus.tech, vdsa@sarus.tech
}
\fi

\begin{document}

\maketitle

\begin{abstract}
AAAI creates proceedings, working notes, and technical reports directly from electronic source furnished by the authors. To ensure that all papers in the publication have a uniform appearance, authors must adhere to the following instructions.
\end{abstract}

\section*{Useful links}
\subsection{PPAI}
Last year papers:
\url{https://aaai-ppai23.github.io/#sp2}
This year program:
\url{https://ppai-workshop.github.io/}

\subsection{Comparable open-source projects}

\begin{itemize}
    \item Paszke et al. 2017 - Automatic differentiation in PyTorch \url{https://openreview.net/pdf?id=BJJsrmfCZ}
    \item Frostig et al. 2018 - Compiling machine learning programs via high-level tracing \url{https://mlsys.org/Conferences/2019/doc/2018/146.pdf}
\end{itemize}

\subsection{Comparable DP SQL papers}

\begin{itemize}
    \item Lessons Learned: Surveying the Practicality of Differential Privacy in the Industry \cite{garrido2022lessons}
    \item Tumult Analytics: a robust, easy-to-use, scalable, and expressive framework for differential privacy \cite{berghel2022tumult}
    \item Differentially Private SQL with Bounded User Contribution \cite{wilson2019differentially}
    \item CHORUS: a Programming Framework for Building Scalable Differential Privacy Mechanisms \cite{johnson2020chorus}
    \item Towards Practical Differential Privacy for SQL Queries \cite{johnson2018towards}
\end{itemize}

\section{Introduction}

In recent years, the importance of safeguarding privacy when dealing with personal data has continuously increased.
Traditional anonymization techniques have proven vulnerable to re-identification, as demonstrated by numerous works \cite{archie2018s, dwork2017exposed, narayanan2008robust, sweeney2013identifying}. The total cost of data breaches has also significantly increased \cite{ibm2023cost}. And governments have introduced stricter data protection laws.
Yet, the collection, sharing, and utilization of data hold the potential to generate significant value across various industries, including healthcare, finance, transportation, and energy distribution.

To realize these benefits while managing privacy risks, researchers have turned to differential privacy (DP) \cite{wood2018differential, dwork2014algorithmic}, which has become the gold standard in academia since its introduction by Dwork et al. in 2006 \cite{dwork2006calibrating} due to its provable and automatic privacy guarantees.

Despite the availability of open-source tools, DP adoption remains limited. This 
In summary, our main contributions are as follows:

Motivation DP
Solutions existantes
Problème non résolu et nécessité de Qrlew



\section{Preparing an Anonymous Submission}

This document details the formatting requirements for anonymous submissions. The requirements are the same as for camera ready papers but with a few notable differences:

\begin{itemize}
    \item Anonymous submissions must not include the author names and affiliations. Write ``Anonymous Submission'' as the ``sole author'' and leave the affiliations empty.
    \item The PDF document's metadata should be cleared with a metadata-cleaning tool before submitting it. This is to prevent leaked information from revealing your identity.
    \item References must be anonymized whenever the reader can infer that they are to the authors' previous work.
    \item AAAI's copyright notice should not be included as a footer in the first page.
    \item Only the PDF version is required at this stage. No source versions will be requested, nor any copyright transfer form.
\end{itemize}

You can achieve all of the above by enabling the \texttt{submission} option when loading the \texttt{aaai24} package:

\begin{quote}\begin{scriptsize}\begin{verbatim}
\documentclass[letterpaper]{article}
\usepackage[submission]{aaai24}
\end{verbatim}\end{scriptsize}\end{quote}

The remainder of this document are the original camera-
ready instructions. Any contradiction of the above points
ought to be ignored while preparing anonymous submis-
sions.

\section{Camera-Ready Guidelines}

Congratulations on having a paper selected for inclusion in an AAAI Press proceedings or technical report! This document details the requirements necessary to get your accepted paper published using PDF\LaTeX{}. If you are using Microsoft Word, instructions are provided in a different document. AAAI Press does not support any other formatting software.

The instructions herein are provided as a general guide for experienced \LaTeX{} users. If you do not know how to use \LaTeX{}, please obtain assistance locally. AAAI cannot provide you with support and the accompanying style files are \textbf{not} guaranteed to work. If the results you obtain are not in accordance with the specifications you received, you must correct your source file to achieve the correct result.

These instructions are generic. Consequently, they do not include specific dates, page charges, and so forth. Please consult your specific written conference instructions for details regarding your submission. Please review the entire document for specific instructions that might apply to your particular situation. All authors must comply with the following:

\begin{itemize}
\item You must use the 2024 AAAI Press \LaTeX{} style file and the aaai24.bst bibliography style files, which are located in the 2024 AAAI Author Kit (aaai24.sty, aaai24.bst).
\item You must complete, sign, and return by the deadline the AAAI copyright form (unless directed by AAAI Press to use the AAAI Distribution License instead).
\item You must read and format your paper source and PDF according to the formatting instructions for authors.
\item You must submit your electronic files and abstract using our electronic submission form \textbf{on time.}
\item You must pay any required page or formatting charges to AAAI Press so that they are received by the deadline.
\item You must check your paper before submitting it, ensuring that it compiles without error, and complies with the guidelines found in the AAAI Author Kit.
\end{itemize}

\section{Copyright}
All papers submitted for publication by AAAI Press must be accompanied by a valid signed copyright form. They must also contain the AAAI copyright notice at the bottom of the first page of the paper. There are no exceptions to these requirements. If you fail to provide us with a signed copyright form or disable the copyright notice, we will be unable to publish your paper. There are \textbf{no exceptions} to this policy. You will find a PDF version of the AAAI copyright form in the AAAI AuthorKit. Please see the specific instructions for your conference for submission details.

\section{Formatting Requirements in Brief}
We need source and PDF files that can be used in a variety of ways and can be output on a variety of devices. The design and appearance of the paper is strictly governed by the aaai style file (aaai24.sty).
\textbf{You must not make any changes to the aaai style file, nor use any commands, packages, style files, or macros within your own paper that alter that design, including, but not limited to spacing, floats, margins, fonts, font size, and appearance.} AAAI imposes requirements on your source and PDF files that must be followed. Most of these requirements are based on our efforts to standardize conference manuscript properties and layout. All papers submitted to AAAI for publication will be recompiled for standardization purposes. Consequently, every paper submission must comply with the following requirements:

\begin{itemize}
\item Your .tex file must compile in PDF\LaTeX{} --- (you may not include .ps or .eps figure files.)
\item All fonts must be embedded in the PDF file --- including your figures.
\item Modifications to the style file, whether directly or via commands in your document may not ever be made, most especially when made in an effort to avoid extra page charges or make your paper fit in a specific number of pages.
\item No type 3 fonts may be used (even in illustrations).
\item You may not alter the spacing above and below captions, figures, headings, and subheadings.
\item You may not alter the font sizes of text elements, footnotes, heading elements, captions, or title information (for references and mathematics, please see the limited exceptions provided herein).
\item You may not alter the line spacing of text.
\item Your title must follow Title Case capitalization rules (not sentence case).
\item \LaTeX{} documents must use the Times or Nimbus font package (you may not use Computer Modern for the text of your paper).
\item No \LaTeX{} 209 documents may be used or submitted.
\item Your source must not require use of fonts for non-Roman alphabets within the text itself. If your paper includes symbols in other languages (such as, but not limited to, Arabic, Chinese, Hebrew, Japanese, Thai, Russian and other Cyrillic languages), you must restrict their use to bit-mapped figures. Fonts that require non-English language support (CID and Identity-H) must be converted to outlines or 300 dpi bitmap or removed from the document (even if they are in a graphics file embedded in the document).
\item Two-column format in AAAI style is required for all papers.
\item The paper size for final submission must be US letter without exception.
\item The source file must exactly match the PDF.
\item The document margins may not be exceeded (no overfull boxes).
\item The number of pages and the file size must be as specified for your event.
\item No document may be password protected.
\item Neither the PDFs nor the source may contain any embedded links or bookmarks (no hyperref or navigator packages).
\item Your source and PDF must not have any page numbers, footers, or headers (no pagestyle commands).
\item Your PDF must be compatible with Acrobat 5 or higher.
\item Your \LaTeX{} source file (excluding references) must consist of a \textbf{single} file (use of the ``input" command is not allowed.
\item Your graphics must be sized appropriately outside of \LaTeX{} (do not use the ``clip" or ``trim'' command) .
\end{itemize}

If you do not follow these requirements, your paper will be returned to you to correct the deficiencies.

\section{What Files to Submit}
You must submit the following items to ensure that your paper is published:
\begin{itemize}
\item A fully-compliant PDF file.
\item Your \LaTeX{} source file submitted as a \textbf{single} .tex file (do not use the ``input" command to include sections of your paper --- every section must be in the single source file). (The only allowable exception is .bib file, which should be included separately).
\item The bibliography (.bib) file(s).
\item Your source must compile on our system, which includes only standard \LaTeX{} 2020 TeXLive support files.
\item Only the graphics files used in compiling paper.
\item The \LaTeX{}-generated files (e.g. .aux,  .bbl file, PDF, etc.).
\end{itemize}

Your \LaTeX{} source will be reviewed and recompiled on our system (if it does not compile, your paper will be returned to you. \textbf{Do not submit your source in multiple text files.} Your single \LaTeX{} source file must include all your text, your bibliography (formatted using aaai24.bst), and any custom macros.

Your files should work without any supporting files (other than the program itself) on any computer with a standard \LaTeX{} distribution.

\textbf{Do not send files that are not actually used in the paper.} Avoid including any files not needed for compiling your paper, including, for example, this instructions file, unused graphics files, style files, additional material sent for the purpose of the paper review, intermediate build files and so forth.

\textbf{Obsolete style files.} The commands for some common packages (such as some used for algorithms), may have changed. Please be certain that you are not compiling your paper using old or obsolete style files.

\textbf{Final Archive.} Place your source files in a single archive which should be compressed using .zip. The final file size may not exceed 10 MB.
Name your source file with the last (family) name of the first author, even if that is not you.


\section{Using \LaTeX{} to Format Your Paper}

The latest version of the AAAI style file is available on AAAI's website. Download this file and place it in the \TeX\ search path. Placing it in the same directory as the paper should also work. You must download the latest version of the complete AAAI Author Kit so that you will have the latest instruction set and style file.

\subsection{Document Preamble}

In the \LaTeX{} source for your paper, you \textbf{must} place the following lines as shown in the example in this subsection. This command set-up is for three authors. Add or subtract author and address lines as necessary, and uncomment the portions that apply to you. In most instances, this is all you need to do to format your paper in the Times font. The helvet package will cause Helvetica to be used for sans serif. These files are part of the PSNFSS2e package, which is freely available from many Internet sites (and is often part of a standard installation).

Leave the setcounter for section number depth commented out and set at 0 unless you want to add section numbers to your paper. If you do add section numbers, you must uncomment this line and change the number to 1 (for section numbers), or 2 (for section and subsection numbers). The style file will not work properly with numbering of subsubsections, so do not use a number higher than 2.

% Using the \centering command instead of \begin{center} ... \end{center} will save space
% Positioning your figure at the top of the page will save space and make the paper more readable
% Using 0.95\columnwidth in conjunction with the


Your paper must compile in PDF\LaTeX{}. Consequently, all your figures must be .jpg, .png, or .pdf. You may not use the .gif (the resolution is too low), .ps, or .eps file format for your figures.

Figures, drawings, tables, and photographs should be placed throughout the paper on the page (or the subsequent page) where they are first discussed. Do not group them together at the end of the paper. If placed at the top of the paper, illustrations may run across both columns. Figures must not invade the top, bottom, or side margin areas. Figures must be inserted using the \textbackslash usepackage\{graphicx\}. Number figures sequentially, for example, figure 1, and so on. Do not use minipage to group figures.

If you normally create your figures using pgfplots, please create the figures first, and then import them as pdfs with proper bounding boxes, as the bounding and trim boxes created by pfgplots are fragile and not valid.

When you include your figures, you must crop them \textbf{outside} of \LaTeX{}. The command \textbackslash includegraphics*[clip=true, viewport 0 0 10 10]{...} might result in a PDF that looks great, but the image is \textbf{not really cropped.} The full image can reappear (and obscure whatever it is overlapping) when page numbers are applied or color space is standardized. Figures \ref{fig1}, and \ref{fig2} display some unwanted results that often occur.

If your paper includes illustrations that are not compatible with PDF\TeX{} (such as .eps or .ps documents), you will need to convert them. The epstopdf package will usually work for eps files. You will need to convert your ps files to PDF in either case.

\subsubsection {Figure Captions.}The illustration number and caption must appear \textit{under} the illustration. Labels and other text with the actual illustration must be at least nine-point type. However, the font and size of figure captions must be 10 point roman. Do not make them smaller, bold, or italic. (Individual words may be italicized if the context requires differentiation.)

\subsection{Tables}

Tables should be presented in 10 point roman type. If necessary, they may be altered to 9 point type. You may not use any commands that further reduce point size below nine points. Tables that do not fit in a single column must be placed across double columns. If your table won't fit within the margins even when spanning both columns, you must split it. Do not use minipage to group tables.

\subsubsection {Table Captions.} The number and caption for your table must appear \textit{under} (not above) the table.  Additionally, the font and size of table captions must be 10 point roman and must be placed beneath the figure. Do not make them smaller, bold, or italic. (Individual words may be italicized if the context requires differentiation.)



\subsubsection{Low-Resolution Bitmaps.}
You may not use low-resolution (such as 72 dpi) screen-dumps and GIF files---these files contain so few pixels that they are always blurry, and illegible when printed. If they are color, they will become an indecipherable mess when converted to black and white. This is always the case with gif files, which should never be used. The resolution of screen dumps can be increased by reducing the print size of the original file while retaining the same number of pixels. You can also enlarge files by manipulating them in software such as PhotoShop. Your figures should be 300 dpi when incorporated into your document.

\subsubsection{\LaTeX{} Overflow.}
\LaTeX{} users please beware: \LaTeX{} will sometimes put portions of the figure or table or an equation in the margin. If this happens, you need to make the figure or table span both columns. If absolutely necessary, you may reduce the figure, or reformat the equation, or reconfigure the table.{ \bf Check your log file!} You must fix any overflow into the margin (that means no overfull boxes in \LaTeX{}). \textbf{Nothing is permitted to intrude into the margin or gutter.}


\subsubsection{Using Color.}
Use of color is restricted to figures only. It must be WACG 2.0 compliant. (That is, the contrast ratio must be greater than 4.5:1 no matter the font size.) It must be CMYK, NOT RGB. It may never be used for any portion of the text of your paper. The archival version of your paper will be printed in black and white and grayscale. The web version must be readable by persons with disabilities. Consequently, because conversion to grayscale can cause undesirable effects (red changes to black, yellow can disappear, and so forth), we strongly suggest you avoid placing color figures in your document. If you do include color figures, you must (1) use the CMYK (not RGB) colorspace and (2) be mindful of readers who may happen to have trouble distinguishing colors. Your paper must be decipherable without using color for distinction.

\subsubsection{Drawings.}
We suggest you use computer drawing software (such as Adobe Illustrator or, (if unavoidable), the drawing tools in Microsoft Word) to create your illustrations. Do not use Microsoft Publisher. These illustrations will look best if all line widths are uniform (half- to two-point in size), and you do not create labels over shaded areas. Shading should be 133 lines per inch if possible. Use Times Roman or Helvetica for all figure call-outs. \textbf{Do not use hairline width lines} --- be sure that the stroke width of all lines is at least .5 pt. Zero point lines will print on a laser printer, but will completely disappear on the high-resolution devices used by our printers.

\subsubsection{Photographs and Images.}
Photographs and other images should be in grayscale (color photographs will not reproduce well; for example, red tones will reproduce as black, yellow may turn to white, and so forth) and set to a minimum of 300 dpi. Do not prescreen images.

\subsubsection{Resizing Graphics.}
Resize your graphics \textbf{before} you include them with LaTeX. You may \textbf{not} use trim or clip options as part of your \textbackslash includegraphics command. Resize the media box of your PDF using a graphics program instead.

\subsubsection{Fonts in Your Illustrations.}
You must embed all fonts in your graphics before including them in your LaTeX document.

\subsubsection{Algorithms.}
Algorithms and/or programs are a special kind of figures. Like all illustrations, they should appear floated to the top (preferably) or bottom of the page. However, their caption should appear in the header, left-justified and enclosed between horizontal lines, as shown in Algorithm~\ref{alg:algorithm}. The algorithm body should be terminated with another horizontal line. It is up to the authors to decide whether to show line numbers or not, how to format comments, etc.

In \LaTeX{} algorithms may be typeset using the {\tt algorithm} and {\tt algorithmic} packages, but you can also use one of the many other packages for the task.

\begin{algorithm}[tb]
\caption{Example algorithm}
\label{alg:algorithm}
\textbf{Input}: Your algorithm's input\\
\textbf{Parameter}: Optional list of parameters\\
\textbf{Output}: Your algorithm's output
\begin{algorithmic}[1] %[1] enables line numbers
\STATE Let $t=0$.
\WHILE{condition}
\STATE Do some action.
\IF {conditional}
\STATE Perform task A.
\ELSE
\STATE Perform task B.
\ENDIF
\ENDWHILE
\STATE \textbf{return} solution
\end{algorithmic}
\end{algorithm}

\subsubsection{Listings.}
Listings are much like algorithms and programs. They should also appear floated to the top (preferably) or bottom of the page. Listing captions should appear in the header, left-justified and enclosed between horizontal lines as shown in Listing~\ref{lst:listing}. Terminate the body with another horizontal line and avoid any background color. Line numbers, if included, must appear within the text column.

\begin{listing}[tb]%
\caption{Example listing {\tt quicksort.hs}}%
\label{lst:listing}%
\begin{lstlisting}[language=Haskell]
quicksort :: Ord a => [a] -> [a]
quicksort []     = []
quicksort (p:xs) = (quicksort lesser) ++ [p] ++ (quicksort greater)
	where
		lesser  = filter (< p) xs
		greater = filter (>= p) xs
\end{lstlisting}
\end{listing}

\subsection{References}
The AAAI style includes a set of definitions for use in formatting references with BibTeX. These definitions make the bibliography style fairly close to the ones  specified in the Reference Examples appendix below. To use these definitions, you also need the BibTeX style file ``aaai24.bst," available in the AAAI Author Kit on the AAAI web site. Then, at the end of your paper but before \textbackslash end{document}, you need to put the following lines:

\begin{quote}
\begin{small}
\textbackslash bibliography\{bibfile1,bibfile2,...\}
\end{small}
\end{quote}

Please note that the aaai24.sty class already sets the bibliographystyle for you, so you do not have to place any \textbackslash bibliographystyle command in the document yourselves. The aaai24.sty file is incompatible with the hyperref and navigator packages. If you use either, your references will be garbled and your paper will be returned to you.

References may be the same size as surrounding text. However, in this section (only), you may reduce the size to \textbackslash small if your paper exceeds the allowable number of pages. Making it any smaller than 9 point with 10 point linespacing, however, is not allowed. A more precise and exact method of reducing the size of your references minimally is by means of the following command: \begin{quote}
\textbackslash fontsize\{9.8pt\}\{10.8pt\}
\textbackslash selectfont\end{quote}

\noindent You must reduce the size equally for both font size and line spacing, and may not reduce the size beyond \{9.0pt\}\{10.0pt\}.

The list of files in the \textbackslash bibliography command should be the names of your BibTeX source files (that is, the .bib files referenced in your paper).

The following commands are available for your use in citing references:
\begin{quote}
{\em \textbackslash cite:} Cites the given reference(s) with a full citation. This appears as ``(Author Year)'' for one reference, or ``(Author Year; Author Year)'' for multiple references.\smallskip\\
{\em \textbackslash shortcite:} Cites the given reference(s) with just the year. This appears as ``(Year)'' for one reference, or ``(Year; Year)'' for multiple references.\smallskip\\
{\em \textbackslash citeauthor:} Cites the given reference(s) with just the author name(s) and no parentheses.\smallskip\\
{\em \textbackslash citeyear:} Cites the given reference(s) with just the date(s) and no parentheses.
\end{quote}
You may also use any of the \emph{natbib} citation commands.


\section{Proofreading Your PDF}
Please check all the pages of your PDF file. The most commonly forgotten element is the acknowledgements --- especially the correct grant number. Authors also commonly forget to add the metadata to the source, use the wrong reference style file, or don't follow the capitalization rules or comma placement for their author-title information properly. A final common problem is text (expecially equations) that runs into the margin. You will need to fix these common errors before submitting your file.

\section{Improperly Formatted Files }
In the past, AAAI has corrected improperly formatted files submitted by the authors. Unfortunately, this has become an increasingly burdensome expense that we can no longer absorb). Consequently, if your file is improperly formatted, it will be returned to you for correction.

\section{Naming Your Electronic File}
We require that you name your \LaTeX{} source file with the last name (family name) of the first author so that it can easily be differentiated from other submissions. Complete file-naming instructions will be provided to you in the submission instructions.

\section{Submitting Your Electronic Files to AAAI}
Instructions on paper submittal will be provided to you in your acceptance letter.

\section{Inquiries}
If you have any questions about the preparation or submission of your paper as instructed in this document, please contact AAAI Press at the address given below. If you have technical questions about implementation of the aaai style file, please contact an expert at your site. We do not provide technical support for \LaTeX{} or any other software package. To avoid problems, please keep your paper simple, and do not incorporate complicated macros and style files.

\begin{quote}
\noindent AAAI Press\\
1900 Embarcadero Road, Suite 101\\
Palo Alto, California 94303-3310 USA\\
\textit{Telephone:} (650) 328-3123\\
\textit{E-mail:} See the submission instructions for your particular conference or event.
\end{quote}

\section{Additional Resources}
\LaTeX{} is a difficult program to master. If you've used that software, and this document didn't help or some items were not explained clearly, we recommend you read Michael Shell's excellent document (testflow doc.txt V1.0a 2002/08/13) about obtaining correct PS/PDF output on \LaTeX{} systems. (It was written for another purpose, but it has general application as well). It is available at www.ctan.org in the tex-archive.

\appendix
\section{Reference Examples}
\label{sec:reference_examples}

\vspace{.2em}
For the most up to date version of the AAAI reference style, please consult the \textit{AI Magazine} Author Guidelines at \url{https://aaai.org/ojs/index.php/aimagazine/about/submissions#authorGuidelines}

\section{Acknowledgments}
AAAI is especially grateful to Peter Patel Schneider for his work in implementing the original aaai.sty file, liberally using the ideas of other style hackers, including Barbara Beeton. We also acknowledge with thanks the work of George Ferguson for his guide to using the style and BibTeX files --- which has been incorporated into this document --- and Hans Guesgen, who provided several timely modifications, as well as the many others who have, from time to time, sent in suggestions on improvements to the AAAI style. We are especially grateful to Francisco Cruz, Marc Pujol-Gonzalez, and Mico Loretan for the improvements to the Bib\TeX{} and \LaTeX{} files made in 2020.

The preparation of the \LaTeX{} and Bib\TeX{} files that implement these instructions was supported by Schlumberger Palo Alto Research, AT\&T Bell Laboratories, Morgan Kaufmann Publishers, The Live Oak Press, LLC, and AAAI Press. Bibliography style changes were added by Sunil Issar. \verb+\+pubnote was added by J. Scott Penberthy. George Ferguson added support for printing the AAAI copyright slug. Additional changes to aaai24.sty and aaai24.bst have been made by Francisco Cruz and Marc Pujol-Gonzalez.

\bigskip
\noindent Thank you for reading these instructions carefully. We look forward to receiving your electronic files!

% \bibliography{aaai24}
\bibliography{qrlew}

\end{document}
