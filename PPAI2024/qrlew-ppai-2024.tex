%File: anonymous-submission-latex-2024.tex
\documentclass[letterpaper]{article} % DO NOT CHANGE THIS
\usepackage[submission]{aaai24}  % DO NOT CHANGE THIS
\usepackage{times}  % DO NOT CHANGE THIS
\usepackage{helvet}  % DO NOT CHANGE THIS
\usepackage{courier}  % DO NOT CHANGE THIS
\usepackage[hyphens]{url}  % DO NOT CHANGE THIS
\usepackage{graphicx} % DO NOT CHANGE THIS
\urlstyle{rm} % DO NOT CHANGE THIS
\def\UrlFont{\rm}  % DO NOT CHANGE THIS
\usepackage{natbib}  % DO NOT CHANGE THIS AND DO NOT ADD ANY OPTIONS TO IT
\usepackage{caption} % DO NOT CHANGE THIS AND DO NOT ADD ANY OPTIONS TO IT
\frenchspacing  % DO NOT CHANGE THIS
\setlength{\pdfpagewidth}{8.5in} % DO NOT CHANGE THIS
\setlength{\pdfpageheight}{11in} % DO NOT CHANGE THIS
%
% These are recommended to typeset algorithms but not required. See the subsubsection on algorithms. Remove them if you don't have algorithms in your paper.
\usepackage{algorithm}
\usepackage{algorithmic}

%
% These are are recommended to typeset listings but not required. See the subsubsection on listing. Remove this block if you don't have listings in your paper.
\usepackage{newfloat}
\usepackage{listings}
\DeclareCaptionStyle{ruled}{labelfont=normalfont,labelsep=colon,strut=off} % DO NOT CHANGE THIS
\lstset{%
	basicstyle={\footnotesize\ttfamily},% footnotesize acceptable for monospace
	numbers=left,numberstyle=\footnotesize,xleftmargin=2em,% show line numbers, remove this entire line if you don't want the numbers.
	aboveskip=0pt,belowskip=0pt,%
	showstringspaces=false,tabsize=2,breaklines=true}
\floatstyle{ruled}
\newfloat{listing}{tb}{lst}{}
\floatname{listing}{Listing}
%
% Keep the \pdfinfo as shown here. There's no need
% for you to add the /Title and /Author tags.
\pdfinfo{
/TemplateVersion (2024.1)
}

\setcounter{secnumdepth}{0} %May be changed to 1 or 2 if section numbers are desired.

% The file aaai24.sty is the style file for AAAI Press
% proceedings, working notes, and technical reports.
%

% A command for the project name
\newcommand{\qrlew}{\emph{Qrlew}}

% Title

% Your title must be in mixed case, not sentence case.
% That means all verbs (including short verbs like be, is, using,and go),
% nouns, adverbs, adjectives should be capitalized, including both words in hyphenated terms, while
% articles, conjunctions, and prepositions are lower case unless they
% directly follow a colon or long dash
\title{\qrlew: Differentially Privacte SQL Query Rewriting}
\author{
    %Authors
    % Authors
    Nicolas Grislain\textsuperscript{\rm 1}
    Paul Roussel\textsuperscript{\rm 1}
    Victoria de Sainte-Agathe\textsuperscript{\rm 1}
}
\affiliations{
    %Afiliations
    \textsuperscript{\rm 1}Sarus Technologies\\
    % If you have multiple authors and multiple affiliations
    % use superscripts in text and roman font to identify them.
    % For example,

    % Sunil Issar\textsuperscript{\rm 2},
    % J. Scott Penberthy\textsuperscript{\rm 3},
    % George Ferguson\textsuperscript{\rm 4},
    % Hans Guesgen\textsuperscript{\rm 5}
    % Note that the comma should be placed after the superscript

    1900 Embarcadero Road, Suite 101\\
    Palo Alto, California 94303-3310 USA\\
    % email address must be in roman text type, not monospace or sans serif
    proceedings-questions@aaai.org
%
% See more examples next
}

\iffalse
%Example, Multiple Authors, ->> remove \iffalse,\fi and place them surrounding AAAI title to use it
\title{\qrlew: automatic differential privacy for SQL queries}
\author {
    % Authors
    Nicolas Grislain,
    Paul Roussel,
    Victoria de Sainte-Agathe,
}
\affiliations {
    % Affiliations
    ng@sarus.tech, pr@sarus.tech, vdsa@sarus.tech
}
\fi

\begin{document}

\maketitle

\begin{abstract}
AAAI creates proceedings, working notes, and technical reports directly from electronic source furnished by the authors. To ensure that all papers in the publication have a uniform appearance, authors must adhere to the following instructions.
\end{abstract}

\section{Introduction}

In recent years, the importance of safeguarding privacy when dealing with personal data has continuously increased.
Traditional anonymization techniques have proven vulnerable to re-identification, as demonstrated by numerous works \cite{archie2018s, dwork2017exposed, narayanan2008robust, sweeney2013identifying}.
The total cost of data breaches has also significantly increased \cite{ibm2023cost} and governments have introduced stricter data protection laws.
Yet, the collection, sharing, and utilization of data hold the potential to generate significant value across various industries, including healthcare, finance, transportation, and energy distribution.

To realize these benefits while managing privacy risks, researchers have turned to \emph{differential privacy (DP)} \cite{wood2018differential, dwork2014algorithmic}, which has become the gold standard in academia since its introduction by Dwork et al. in 2006 \cite{dwork2006calibrating} due to its provable and automatic privacy guarantees.

Despite the availability of open-source tools, DP adoption remains limited.
One of the reasons for this lack of adoption is the relative complexity of the existing tools considered the utility of the results.
\qrlew{} has been designed to solve these problems by providing the following features:
\begin{description}
    \item[\qrlew{} provides automatic output privacy guarantees] With \qrlew{} a \emph{data owner} can let an analyst (\emph{data practitionner}) with no expertise in privacy protection run arbitrary SQL queries with strong privacy garantees on the output.
    \item[\qrlew{} leverages existing infrastructures] \qrlew{} rewrites a SQL query into a \emph{differentially private} SQL query that can be run on any data-store with a SQL interface from lightweight DB to big-data stores.
This removes the need for a custom execution engine and enables \emph{differentially private analytics with virtually no technical integration}.
    \item[\qrlew{} leverages synthetic data]. Synthetic data are an increasingly popular way of \emph{privatizing} a dataset. Using jointly \emph{differentially private} mechanisms and \emph{differentially private} synthetic data can be a simple, yet powerful, way of managing a privacy budget and reaching better utility-privacy tradeoffs.
\end{description}

% Motivation DP
% Solutions existantes
% Problème non résolu et nécessité de \qrlew

% à développer plus:
% - adaptativité -> utiliser output DP pour tuner DP aval
% - More mechanisms
% - Interconnectivité

\section{Assumptions and Design Goals}

In this work, we assume the \emph{central model of differential privacy} \cite{near2020threat}, where a trusted central organization: Hospital, Insurance Company, Utility Provider, called the \emph{data owner}, collects and stores personal data in a secured database. and whishes to let untrusted \emph{data-practitionners} run SQL queries on its data.
Furthermore, the 
\qrlew{} was designed to ease the 

\section{General architecture}

In this work, we assume the \emph{central model of differential privacy} \cite{near2020threat}, where a trusted central organization: Hospital, Insurance Company, Utility Provider, called the \emph{data owner}, collects and stores personal data in a secured database. and whishes to let untrusted \emph{data-practitionners} run SQL queries on its data.
Furthermore, the 
\qrlew{} was designed to ease the 

\section{Paul on compilation}

Rewriting in \qrlew{} refers to the process of altering parts of an SQL query by substituting them with different components to alter the privacy properties of the result. This substitution aims to achieve specific objectives, such as ensuring privacy through the incorporation of differential privacy mechanisms. To facilitate this work, we decompose the SQL query in the form of a computation graph where each node (a \texttt{Relation}) is the result of transformations representing part of the SQL queries.

The main goals of the differential privacy rewriting are to modify SQL queries to ensure compliance with differential privacy frameworks, protecting sensitive data, and to guarantee that these modifications are consistent and deterministic, adhering to established privacy standards.

The challenge lies in accurately identifying sections for modification across a complex array of potential transformations, vigilantly tracking the integrity of data rows tied to protected entities during extensive aggregations, and adeptly applying the correct rewriting rules tailored to the relation dynamics of the query, such as Joins, Maps, or Reduces, while considering the original configuration of the sensitive data.

\subsection{Rewriting}

\section{Victoria on DP mech and DP test}

\section{Comparison to other systems}

\section{Known limitations}

\qrlew{} relies on the random number generator of the SQL engine used. It is usually not a cryptographic noise.

\qrlew{} uses the floating-point numbers of the host SQL engine, therefore our system is liable to the vulnerabilities described in 


\section*{Useful links}
\subsection{PPAI}
Last year papers:
\url{https://aaai-ppai23.github.io/#sp2}
This year program:
\url{https://ppai-workshop.github.io/}

\subsection{Comparable open-source projects}

\begin{itemize}
    \item Paszke et al. 2017 - Automatic differentiation in PyTorch \url{https://openreview.net/pdf?id=BJJsrmfCZ}
    \item Frostig et al. 2018 - Compiling machine learning programs via high-level tracing \url{https://mlsys.org/Conferences/2019/doc/2018/146.pdf}
\end{itemize}

\subsection{Comparable DP SQL papers}

\begin{itemize}
    \item Lessons Learned: Surveying the Practicality of Differential Privacy in the Industry \cite{garrido2022lessons}
    \item Tumult Analytics: a robust, easy-to-use, scalable, and expressive framework for differential privacy \cite{berghel2022tumult}
    \item Differentially Private SQL with Bounded User Contribution \cite{wilson2019differentially}
    \item CHORUS: a Programming Framework for Building Scalable Differential Privacy Mechanisms \cite{johnson2020chorus}
    \item Towards Practical Differential Privacy for SQL Queries \cite{johnson2018towards}
\end{itemize}

\bigskip
\noindent Thank you for reading these instructions carefully. We look forward to receiving your electronic files!

% \bibliography{aaai24}
\bibliography{qrlew}

\end{document}
